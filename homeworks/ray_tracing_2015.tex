\documentclass{article} 
\usepackage[noend]{algpseudocode}
\usepackage{amsmath}
\usepackage{amssymb}
\usepackage{algorithm}
\usepackage{graphicx}
\usepackage{hyperref}
\usepackage[margin=1.25in]{geometry}

\title{MPCS 51087 -- Homework 4}
%\author{Ron Rahaman}

\begin{document}
\maketitle

\section{Introduction to Ray Tracing}

Ray tracing is a powerful method for rendering three-dimensional objects with
complex light interactions.  Many other physical phenomena are also analgous to
ray tracing -- for example, simulating neutrons passing through matter.  Figure
1 shows the basic idea for a reflective object.  An observer (the
eyeball) is viewing an object through a window (the rectangle).  The object is
illuminated by a light source, which is emitting rays of light (the arrows) in
many directions.  The observer will see rays that reflect off of the object.

\begin{figure}
  \label{FigBasic}
  \centering
  \includegraphics[width=0.4\textwidth]{RTfigure3.jpg}
  \caption{Illustration of Ray-Tracing.  Image credit: [\ref{ItemUNC}]}
\end{figure}

Our task is to render the image seen by the observer through the window. To do
so, the simulation can do one or both of the following:
\begin{itemize}
  \item Simulate light rays starting at the light source and ending at the observer.
  \item Simulate light rays starting from the observer and going backwards to the light source.
\end{itemize}

We will implement the latter scheme in this assignment.  We will simulate a
reflective sphere in black-and-white, illuminated by a single light source.  A
serial version can be implemented in about 100 lines of code.

\section{Vector Notation}

Solving the problem involves the use of 3D vectors.  A vector $\vec V$ has
scalar components,$\vec V = \left(V_X, V_Y, V_Z\right)$. Adding or subtracting
two vectors yields a vector, $\vec V - \vec U = (V_X - U_X, V_Y-U_Y, V_Z-U_Z)$.
Multipying a scalar and a vector yields a scalar, $t  \vec V = \left(tV_X,
tV_Y, tV_Z\right)$.  Dividing a vector by a scalr yields a scalar, $\vec V / t
= (V_X/t, V_Y/t, V_Z/t)$. The dot product (a multiplication between two
vectors) yields a scalar, $\vec V \cdot \vec U = V_X U_X + V_Y U_Y + V_Z U_Z$.
The norm (or magnitude or length) of the vector is a scalar defined by $ | \vec
V | = \sqrt{V_X^2 + V_Y^2 + V_Z^2}$.

\newpage

\section{Theory}

This section describes the theory used in our implementation.  The
implmentation itself is brief and is described in Algorithm 1 below, so
you may skip this section as needed.

Figure 2 shows a 2D cross-section of the problem with the
information we need.  The observer is at the origin and faces the positive
y-direction.  The sphere is located at $\vec C = (C_X, C_Y, C_Z)$ and has
radius $R$. The window is parallel to the (x,z)-plane at $y=W_Y$ and has bounds
$-W_{max} < W_X < W_{max}$ and $-W_{max} < W_Z < W_{max}$.  In the simulation,
we will represent the window as an $n \times n$ grid, $G$. The light source
is located at $\vec L = (L_X, L_Y, L_Z)$.

Our task is to simulate many rays originating from the observer (so-called
``view rays'') with randomly-selected directions.  The steps are as follows:

\begin{figure}
  \label{FigComplex}
  \centering
  \includegraphics[width=\textwidth]{ray_tracing_diagram.pdf}
  \caption{2D Diagram for Ray-Tracing}
\end{figure}

\begin{itemize}

  \item \textbf{Select the direction of view ray ($\vec V$).} Let $\vec V$ be a
    unit vector representing the direction of the view ray.  In spherical
    coordinates, We will randomly select its component angles $\left(\theta,
    \phi\right)$ such that $0 < \theta < \pi$ and $0 < \phi < \pi$.  Then we
    will get $\vec V$ in Cartesion coordinates:
    \begin{align*}
      V_X &= \sin \theta \cos \phi \\
      V_Y &= \sin \theta \sin \phi \\
      V_Z &= \cos \theta
    \end{align*}

  \item \textbf{Find the intersection of the view ray with the window ($\vec
    W$).} Knowing that the window is at $W_Y$, the window's point-of-intersection with
    the view ray is given by the vector $\vec W$:
    $$ \vec W = \frac{W_Y}{V_Y} \vec V $$
    If the view ray is outside the window ($\left|W_X\right| <  W_{max}$ or
    $\left|W_Z\right| < W_{max}$), we reject it and chose a new $\vec V$.


  \item \textbf{Find the intersection of view ray with sphere ($\vec I$).}  Let
    $\vec I$ be the sphere's point-of-intersection with the view ray.  To find
    $\vec I$, we solve the following system of equations:
    \begin{align*}
      \vec I &= t \vec V \\
      \left| \vec I - \vec C \right|^2 &= R^2
    \end{align*}
    These are the equations of the view ray and the sphere, respectively.
    Solving for $t$ yields:
    $$ t = \left(\vec V \cdot \vec C \right) -
    \sqrt{\left(\vec V \cdot \vec C \right)^2 + R^2 - \vec C \cdot \vec C} $$
    which can be back-substituted to get $\vec I$.  If $t$ does not have a real
    solution ($(\vec V \cdot \vec C )^2 + R^2 - \vec C \cdot \vec C < 0$), then
    view ray does not intersect the sphere and we choose a new $\vec V$.

  \item \textbf{Find the observed brightness of the sphere ($b$).}  Next, we want to
    find the brightness of the sphere that is observed at $\vec I$.  To do so , we:

    \begin{itemize}

      \item \textit{Find the unit normal vector ($\vec N$).} The unit normal
        vector $\vec N$ is perpindicular to the sphere's surface at $\vec I$.
        $$ \vec N = \frac{\vec I - \vec C}{\left| \vec I - \vec C \right|} $$

      \item \textit{Find the direction to the light source ($\vec S$).} The
        direction to light source (somtimes called the ``shadow ray'') is
        represented by the unit vector $\vec S$.
        $$ \vec S = \frac{\vec L - \vec I}{\left| \vec L - \vec I \right|} $$

      \item \textit{Find the brightness ($b$).} The brightness can be found
        from $\vec S$ and $\vec N$ using ``Lambertian shading''.
        $$
        b = \left\{
          \begin{array}{ll}
            0 & \quad  \vec S \cdot \vec N < 0 \\
            \vec S \cdot \vec N & \quad  \vec S \cdot \vec N \ge 0 \\
          \end{array}
          \right.
        $$

    \end{itemize}

  \item \textbf{Add the brightness to the window's grid.}
    We find $\left(i,j\right)$ such that $\vec G(i,j)$ is the position of $\vec
    W$ on the window's grid $G$ and let:
    $$ G(i,j) = G(i,j) + b $$

\end{itemize}

\newpage

\section{Implementation}

Algorithm 1 describes a ray-tracing implementation.  As described above, the
observer is at the origin and is facing the positive-$y$ direction.  The sphere
is located at $\vec C = (C_X, C_Y, C_Z)$ and has radius $R$. The light source
is located at $\vec L = (L_X, L_Y, L_Z)$.  The window is parallel to the
(x,z)-plane at $y=W_Y$, has bounds $-W_{max} < W_X < W_{max}$ and $-W_{max} <
W_Z < W_{max}$, and is repsresented by an $n \times n$ grid, $G$.

\begin{algorithm}
  \caption{Ray Tracing Algorithm}
  \label{Alg}
  \begin{algorithmic}[1]
    \State allocate $G[1\ldots n][1\ldots n]$
      \Comment The window is respresented on the grid $G$
    \State $G[i][j] = 0$ for all $(i,j)$
    \For{$n = 1 \ldots N_{rays}$}
    \Repeat
      \State $\theta = $ a random decimal between $\left(0, \pi\right)$
        \Comment The direction of the view ray in spherical coords
      \State $\phi = $ a random decimal between $\left(0, \pi\right)$
      \State $\vec V = \left( \sin \theta \cos \phi, \sin \theta \sin \phi, \cos \theta \right)$
        \Comment The direction of the view ray in Cartesian coords
      \State $ \vec W = \frac{W_Y}{V_Y} \vec V $
        \Comment The intersection of the view ray and the window
    \Until{$\left|W_X\right| >  W_{max} \textbf{ and } \left|W_Z\right| > W_{max} \textbf{ and } \left(\vec V \cdot \vec C \right)^2 + R^2 - \vec C \cdot \vec C < 0$}
    \State $ t = \left(\vec V \cdot \vec C \right) - \sqrt{\left(\vec V \cdot \vec C \right)^2 + R^2 - \vec C \cdot \vec C} $
    \State $\vec I = t \vec V$
      \Comment The intersection of the view ray and the sphere
    \State $ \vec N = \frac{\vec I - \vec C}{\left| \vec I - \vec C \right|} $
      \Comment The unit normal vector at $\vec I$
    \State $\vec S = \frac{\vec L - \vec I}{\left| \vec L - \vec I \right|} $
      \Comment The direction of the light source at $\vec I$
    \State $b = \textsc{max}\left( 0, \vec S \cdot \vec N \right)$
      \Comment The brightness observed at $\vec I$
    \State find $\left(i,j\right)$ such that $G(i,j)$ is the gripoint of $\vec W$ on $G$
    \State $G\left(i,j\right) = G\left(i,j\right) + b$
      \Comment Add brightness to grid
    \EndFor
  \end{algorithmic}
\end{algorithm}

Figure \ref{FigResults} shows a sample image where $\vec L =
(4,4,-1)$, $W_Y = 10$, $W_{max} = 10$, $C = (0,12,0)$, $R = 6$.


\begin{figure}[hb]
  \label{FigResults}
  \centering
  \includegraphics[width=0.4\textwidth]{sphere_render.pdf}
  \caption{Ray-Traced Render of a Sphere Illuminated from Top Left}
\end{figure}

\newpage

\section{Questions}

\begin{enumerate}

  \item Make and test a serial version of Algorithm 1.  Take user input for the
    number of rays and gridpoints; the other parameters can be hardcoded.
    (Hint: Define structs and functions for the vectors and vector operations.
    With these simple abstractions, the code can be written in about 100
    lines.)

  \item Use OpenMP to parallelize Algorithm 1.  Note that, while the
    rays are independent, updating the window could result in write conflicts.
    Therefore, some syncronization is necessary.   Implement two versions with:
    \begin{enumerate}
      \item Atomic additions on a single copy of the window
      \item Threadprivate copies of the window, accumulated into a single
        copy at the end of the simulation
    \end{enumerate}

  \item For both of your synchronization schemes, perform strong-scaling
    analysis.  Compare the performance of the two schemes on a graph.  Was one
    scheme faster than the other?  Why or why not?

  \item Show a sample image produced by any of your ray-tracing implmenentations.

\end{enumerate}

\section{References}
\begin{enumerate}

  \item \label{ItemUNC} Rademacher, Paul.  \textit{Ray Tracing: Graphics for
    the Masses.} \url{https://www.cs.unc.edu/~rademach/xroads-RT/RTarticle.html}

  \item \label{ItemWiki} \textit{Ray tracing (graphics).}
    \url{http://en.wikipedia.org/wiki/Ray_tracing_(graphics)}

\end{enumerate}

\end{document}
